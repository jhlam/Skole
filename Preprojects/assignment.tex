\section{Assignment}
"Information diffusion is a field of network research where a message, starting at a set of seed nodes, is propagated through the edges in a graph according to a simple model. Simulations are used to measure the coverage and speed of the diffusion and are useful in modeling a variety of phenomena such as the spread of disease, memes on the Internet, viral marketing and emergency messages in disaster scenarios.

The effectiveness of a given spreading model is dependent on the initially infected nodes, or seeds. Seed selection for an optimal spread is an NP hard problem and is normally approximated by selecting high-degree nodes or using heuristic methods such as discount-degree or choosing nodes at different levels of the k-core.

This project will explore the feasibility of hardware accelerated seed selection in large graphs as a variation of breadth-first search (BFS) where the decision to visit and infect a child node relies on the outcome of a coin flip. The student is expected to conduct a thorough review of the literature on seed selection algorithms, diffusion models, with emphasis on parallelisation and hardware acceleration and cache models to reduce memory bottlenecks. Hardware design and simulations are possible if time permits."