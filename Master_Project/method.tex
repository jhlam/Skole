\chapter{Method} \label{methode}
- Start by creating pseudocode to illustrate how the Independent cascade model works and how to set up the code.
- Implement the algorithm(ICM) in C code, since High level synthesis synthesis C/C++ code. 
- Create test to make sure the code is correct. (Test early to catch problem early.)
- Port the code over to be high level synthesis compatiable. () 
- analyse the implementation.
- implement on Hardware.
- might want to implement wiwth interface, AXI 4 and stuff like that 


NOTES: 
For our example, a AXI4 Stream would be best i think.

implemented a 16-bit LFSR(linear feedback shift register) to generate the random number, which is used to calculate if the activation will take place.
each have a [5-25\%] activation rate. implemented some different models, use a array instead of a matrix. 

WE have created a generator following the R-mat generator for sparse graph for testing.
We implemented the simulation as a sparse matrix vector multiplication
we implemented that in vivado HLS. and synthesised it
We changed the implementation to accomodate the requirements for HLS; - no recursion, unknown matrix length and allocation of memory. 



pseudocode section:
matrix_vector_multiplication(matrix, x\_vector, result\_vector, coin\_toss, total_node):

	for(row->total_node) do:
		for(col -> total_node) do:
			if(x_vector[col] && matrix[row][col]) do:
				if(row=col) do :
					local\_result = 1;
				else do:
					local\_result = coin\_toss && local_result;
		}
		result[row] = local\_result;
		local\_result = 0;


\section{Parallization}
The function matrix\_vector\_multiplication() performes a single matrix vector multiplication. From the pseudocode, we can see that there is a room for parallization of the SPmV. The outer for loop from the pseudocode, can be parlized since thatfoor loop is not dependent on the variable from the inner for-loop. 

another paralization is during the simulation, after the spmv, the frontier needs to be calculated. And a converged() function is called in the end to determen if the simulation is finished. The frontier calculation and converged can pe run in parallel. 

THe IP-core (currently) is only using around 2\% of the resource available on the FPGA(Zedboard). This gives us a lot of room for parallization of different core. Implemented 2 bus, one for input stream, one for output stream. The output stream consist of the result_vector

THe input vector is the seed_vector and the matrix. The seed and the global probability is used a AXI4 lite since they only read once, 


IP Core Structure:
For the second option, where the RNG core is not implemented into the IP CORE, we would have to have teh random number set as AXI4 stream from the buffer, since we would need to call a stream of random number from the core.


(global vs local probability)
For this experiment, we have set a standard global probability, that signef that each activation of nodes is at a specific probability. THe local probability is that there are a loccal probability to each edge. Each edge can have a specific probability.


\section{High Level Synthesis}
HLS have many predefined protocol, axi4lite, srteam etc etc. These can be implemented in vivado HLS easily. The program needs to be predefined so that it knows when the streaming is done. The axi stream is best suited for stream of data. The work flow to the HLS is to first generate the C or C++ code. Then the create a testbench. then run the co-simulation. The co-simulation runs the test on both on the C-code and the simulated core. After that, run the core to the vivado to implement and design the core. 