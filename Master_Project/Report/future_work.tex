\chapter{Future work} \label{futureWork}
something something

Interesting future work would be explore how the different architecture would improve the performance. For this implementation, we used a core that contained the LFSR, an interesting design would be to have one shared RNG for multiple cores that shares the RNG. The LFSSR would contionously generate random numbers, and each core reads from the random number buffer. This design would free up space and might be able to implement multiple cores in the Zedboard. The only problem would be that for each read from the buffer would result in high overhead. The size and rate that the RNG can be genarated will also be a limiting factor.

for this design, there are multiple optimization scheme that this report have not explored. We have not yet designed a more general scheme for computing. 

For this project, a tiny graph was used, refering to graph 500 benchmarks, the smallest used graph is a toygraph, which is on the scale at around $2^26$

The community of the HLS is very active and frequently responds to forum post seeking help.	Not many work that uses HLS[CITATION NEEDED]. Recently was free, used to cost money.


Learn more about HLS so it can be better utilized. 
different scheme to further work:
- implement parallel
- different scheeme, rng on the outside
- Use other data structure.
- larger graph
- compare this solution to other solution, 
- implement more efficient memory storage
- use other storage method since its sparse. 
- would be interesting to gather information on  energy consumption.
- Implement a more general architecture to handle more total nodes.
- If  there would be enough memory on the board, would be interesting to run 50 times on the board and just return the avrage time and coverage
