\chapter{Introduction} \label{intro}

\section{Motivation}
\textit{Information diffusion} is a field of network research where a message, or data, is propagated through a \textit{network} or a \textit{graph}. The message originates from a chosen set of nodes, known as \textit{seed nodes}. These seed nodes passes the message to it's neighbour through the edges and thus propagate the message over the network. There are different models used in Information diffusion, \textit{Independent Cascade Model}, and \textit{Linear Threshold Model}. Information Diffusion can be used to model different phenomena such as the spread of disease, viral marketing, or even spread of viral videos and \textit{"memes"}\cite{InformationDiffusionThroughBlogspace}. The effectiveness of the simulation is measured in the spread and the speed of propagation.

The effectiveness of the simulation is dependent on the choosen seed nodes. By finding influential nodes we can find target  


Both Independent cascade model and the Linear Threshold model have different uses. W




High level synthesis take algorithms implemented in high level programming language such as C and C++ and and $synthesise$ the design to low level designs\citep{52214}. 


\section{Assignment Interpretation}
From the assignement text, these task were choosen as the main focus of this thesis:\\ \hfil \\ \hfil

\textbf{Task 1 \textit{(mandatory)}} Implement Information Diffusion as Sparse matrix vector multiplication, with high level language C.  \\ \hfil \\ \hfil
\textbf{Task 2 \textit{(mandatory)}} Tailor the implementation of Information Diffusion for synthesise with Vivado HLS.   \\ \hfil \\ \hfil
\textbf{Task 3 \textit{(optional)}} Implement said design on a  Zynq FPGA board. \\ \hfil \\ \hfil
\textbf{Task 4 \textit{(optional)}} Extend the system to be able to handle graph in the size of toy graphs(containing $2^{26})$ nodes) \\ \hfil \\ \hfil

